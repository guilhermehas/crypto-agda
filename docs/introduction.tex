\section{Introduction}

\subsection{Context} 

In 1983, David Chaum created ecash \cite{panurach1996money} an anonymous cryptographic eletronic money.
This cryptocurrency use \gls{rsa} blind signatures \cite{chaum1983blind} to spend transactions.
Later, in 1989, David Chaum found an electronic money corporation called DigiCash Inc.
It was declared bankruptcy in 1998.

Adam Back developed a \gls{pow} scheme for spam control, Hashcash \cite{back2002hashcash}.
To send an email, the hash of the content of this email plus a nonce has to have a numerical value
smaller than a defined target.
So, to create a valid email, the sender (miner) has to spend a considerable \gls{cpu} resource on it.
Because hash functions produce practically random values, so the miner has to guess a lot of nonce
values before finding some nonce that makes the hash of the email less than the target value.
This idea is used in Bitcoin proof of work because each block has a nonce guessed by the miner and
the hash of the block has to be less than the target value.

Wei Dai propose b-money \cite{dai1998b} for the first proposal for distributed digital scarcity.
And Hal Finney created Bit Gold \cite{wallace2011rise}, a reusable proof of work for hash cash for
its algorithm of proof of work.

On 31 October 2008, Satoshi Nakamoto registered the website ``bitcoin.org'' and put a link for his
paper \cite{nakamoto2008bitcoin} in a cryptography mailing list.
In January 2009, Nakamoto released the Bitcoin software as an open-source code.
The identity of Satoshi Nakamoto is still unknown.
Since that time, the total market of Bitcoin came to 330 billion dollars in 17 of December of 2018
when its value reached a historic peak of 20 thousand dollars.

Other cryptocurrencies like Ethereum \cite{wood2014ethereum}, Monero \cite{noether2015ring} and
ZCash \cite{hopwood2016zcash} were created after Bitcoin,
but Bitcoin is still the cryptocurrency with the biggest market value.

Ethereum is a cryptocurrency that uses an account model instead of \gls{utxo} used in Bitcoin for its
transaction data structure.
It uses Solidity as its programming language for smart contracts which resembles Javascript,
so it is easier to program in it than in the stack machine programming language of Bitcoin.
Ethereum is now transitioning from proof of work (used in Bitcoin) to proof of stake
which will be the default proof mechanism of Ethereum 2.0 and will be released in
3 of January of 2020.

Monero and ZCash are both cryptocurrencies that focus on fungibility, privacy and decentralization.
Monero uses an obfuscated public ledger, so anyone can send transactions,
but nobody can tell the source, amount or destination.
Zcash uses the concept of zero-knowledge proof called \gls{kzsnark},
which guarantee privacy for its users.
