%%%%%%%%%%%%%%%%%%%%%%%%%%%%%%%%%%%%%%%%%%%
%%% DOCUMENT PREAMBLE %%%
\documentclass[12pt]{article}

\usepackage{agda}
\usepackage{apacite}
\usepackage{catchfilebetweentags}

\usepackage{ucs}
\usepackage{amssymb}
%% \usepackage{bbm}

\usepackage[english]{babel}
\usepackage{url}
\usepackage[utf8x]{inputenc}
\usepackage{amsmath}
\usepackage{graphicx}
\graphicspath{{images/}}
\usepackage{parskip}
\usepackage{fancyhdr}
\usepackage{vmargin}
\setmarginsrb{3 cm}{2.5 cm}{3 cm}{2.5 cm}{1 cm}{1.5 cm}{1 cm}{1.5 cm}

\title{Programming a cryptocurrency in Agda}								

% Title
\author{Guilherme H. A. Silva}						
% Author
\date{15 de Abril de 2019}
% Date

\makeatletter
\let\thetitle\@title
\let\theauthor\@author
\let\thedate\@date
\makeatother

\pagestyle{fancy}
\fancyhf{}
\rhead{\theauthor}
\lhead{\thetitle}
\cfoot{\thepage}
%%%%%%%%%%%%%%%%%%%%%%%%%%%%%%%%%%%%%%%%%%%%
\begin{document}

%%%%%%%%%%%%%%%%%%%%%%%%%%%%%%%%%%%%%%%%%%%%%%%%%%%%%%%%%%%%%%%%%%%%%%%%%%%%%%%%%%%%%%%%%

\begin{titlepage}
	\centering
    \vspace*{0.5 cm}
   % \includegraphics[scale = 0.075]{bsulogo.png}\\[1.0 cm]	% University Logo
   \begin{center}    \textsc{\Large   Fundação Getúlio Vargas}\\[2.0 cm]	\end{center}% University Name
   \textsc{\Large Modelagem Matemática  }\\[0.5 cm]				% Course Code
	\rule{\linewidth}{0.2 mm} \\[0.4 cm]
	{ \huge \bfseries \thetitle}\\
	\rule{\linewidth}{0.2 mm} \\[1.5 cm]
	
	\begin{minipage}{0.4\textwidth}
		\begin{flushleft} \large
		%	\emph{Submitted To:}\\
		%	Name\\
          % Affiliation\\
           %contact info\\
			\end{flushleft}
			\end{minipage}~
			\begin{minipage}{0.4\textwidth}
            
			\begin{flushright} \large
        \emph{Student:} \\
        Guilherme Horta Alvares da Silva \\
        \emph{Professor:} \\
        Doctor Flávio Codeço Coelho
		\end{flushright}
           
	\end{minipage}\\[2 cm]

  \includegraphics[scale = 0.5]{imgs/FGV.png}
    
	
\end{titlepage}

%%%%%%%%%%%%%%%%%%%%%%%%%%%%%%%%%%%%%%%%%%%%%%%%%%%%%%%%%%%%%%%%%%%%%%%%%%%%%%%%%%%%%%%%%

\tableofcontents
\pagebreak

%%%%%%%%%%%%%%%%%%%%%%%%%%%%%%%%%%%%%%%%%%%%%%%%%%%%%%%%%%%%%%%%%%%%%%%%%%%%%%%%%%%%%%%%%
\renewcommand{\thesection}{\arabic{section}}
\section{Introduction}

\subsection{Cryptocurrencies}

In 1983, David Chaum created ecash \cite{panurach1996money} , an anonymous cryptographic eletronic money.
This cryptocurrency use RSA blind signatures \cite{chaum1983blind} to spend transactions.
Later, in 1989, David Chaum found an eletronic money corporation called DigiCash Inc.
It was declared bankruptcy in 1998.

Adam Back developed a proof-of-work (PoW) scheme for spam control, Hashcash \cite{back2002hashcash}.
To send an email, the hash of the content of this email plus a nounce has to have a numerically value smaller than a defined target.
So, to create a valide email, the sender (miner) has to spend a considerable CPU resource on it.
Because, hash functions produces pratically random values, so the miner has to guess a lot of nounce values before find some nounce that make the hash of the email less than the target value.
This idea is the same that is used in Bitcoin proof of work, because each block has a nounce guessed by the miner and the hash of the block has to be less than the target value.

Wei Dai propose b-money \cite{dai1998b} for the first proposal for distributed digital scarcity.
And Hal Finney created Bit Gold, a reusable proof of work for hashcash for its algorithm of proof of work.

In 31 October 2008, Satoshi Nakamoto registered the website ``bitcoin.org'' and put a link for his paper \cite{nakamoto2008bitcoin} in a cryptography mailing list.
In January 2009, Nakamoto released the bitcoin software as open-source code.
The identity of Satoshi Nakamoto is still unknown.
Since that time, the total market of Bitcoin came to 330 billions dollars in 17 of December of 2018
and his value has a historic record of 20 thousands dollars.

Other cryptocurrencies like Ethereum \cite{wood2014ethereum}, Monero \cite{noether2015ring} and ZCash \cite{hopwood2016zcash} were created after Bitcoin,
but Bitcoin is still the cryptocurrency with the biggest market value.

Ethereum is a cryptocurrency that uses account model instead of UTXO used in bitcoin for its transaction data struct.
It uses Solidity as its programming language for smart contracts, it looks like Java Script,
so it is easier to programming in it than in stack machine programming language of Bitcoin.
Ethereum is now changing from proof of work (used in Bitcoin) to proof of stake.

Monero and ZCash are both cryptocurrency that focus on fungibility, privacy and descentralization.
Monero uses obfuscated public ledger, so anyone can send transcations, but nobody can tell the source, amount or destination.
Zcash uses the concept of zero-knowledge proof called zk-SNARKs, which garantee privacy for its users.

\subsection{Agda Introduction}
Agda is a dependently typed functional language developed by Norell at Chalmers University of Technology as his PhD Thesis.
The current version of Agda is Agda 2.

  \subsection{Martin-Löf type theory}
  Agda is also a proof assistance based on intensional Martin-Löf type theory.

    \subsection{Types}
    In Martin-Löf type theory, there are 3 finites types and 5 types constructors.
    The 0 type contain 0 terms, it is called empty type and it is written bot .

    The 1 type is the type with just 1 canonical term and it represents existence.
    It is called unit type and it is written top .

    The 2 type contains 2 canonical terms. It represents a choice between two values.

    \subsection{Types Constructors}
    The sum-types contain an ordered pair.
    The second type can depend on the first type.
    For example, (n , Vec n).
    So the second element is a vector of size n.

    The pi-types contain functions.
    So given an input type, it will return an output type.
    For Example, Vec n is a type that is constructed by the function type vec.

    In Inductive types, it is a self-referential type.
    For example, linked list of natural numbers, trees, graphs.
    Proofs in inductive types are made by induction.
    For example:
    N-elim : P(0) -> (prod P(n) -> P(S(n))) -> prod P(n)

    Universe types are created to allow proofs written in all types.
    For example, the type of Nat is U0.


It looks like CoQ, but does not have tatics.
Agda is a total language, so it is garanteed that the code always terminal and coverage all inputs.
Agda needs it to be a consistent language.

Agda has inductive data types that are similar to algebric data types in non-depently typed programming language.
The definition of Peano numbers in Agda:

\ExecuteMetaData[latex/agdaExamples.tex]{Nat}

Definitions in Agda are done using induction.
For example, the sum of two numbers in Agda:

\ExecuteMetaData[latex/agdaExamples.tex]{sum}

In Agda, because of dependent types, it is possible to make some restrictions in types that is not possible in other language.
For example, get the first element of a vector.
For it, it is necessary to specify in the type that the vector should have at size greater or equal tha than one.

\ExecuteMetaData[latex/agdaExamples.tex]{vecHead}

Another good example is that in sum of two matrices, they should have the same dimentions.

\ExecuteMetaData[latex/agdaExamples.tex]{matrixSum}

\subsection{UTXO Bitcoin}

There are two kinds of data structures to modeling accounts records and savings states.
The UTXO model used in Bitcoin and the account model used in Ethereum.

  \includegraphics[scale = 0.7]{imgs/account.jpeg}

In account model, it is saved the address and the balance of each address.
For example, the data struct will look like this [(0xabc01, 1.01), (0xabc02, 2.02)].
So the address 0xabc01 has 1.01a of balance and the address 0xabc02 has 2.02 of balance.
In this way, it is possible to easily know how much of balance each address has,
but it is not possible to know how they got in this state.

  \includegraphics[scale = 0.4]{imgs/utxo.png}

In UTXO model, each transaction is saved in the transaction tree.
Every transaction is composed of multiples inputs and multiples outputs.
But all inputs have to never been spent before.

Because of that, in UTXO model, it is easy to make a new transaction from previous one, but it is harder to know how much each one has.
The wallet that calculate how much balance each address has.

In account model, there could be one kind of vulnerability that is less probabable to happen in UTXO model.
Because there is an undesirable intermediary state that there is some address without balance while another has not already received his money.

For example: \\
bobBalance -= 1 \\
Intermediary State \\
aliceBalance += 1

In account model, it is straight foward to know how much balance each address has.
In UTXO model, this calculation is made offchain. It can be a good thing,
because each user has more privacy.

\subsection{TXTree in Agda}

\section{Methods}

\section{Conclusion}

\newpage
 
\bibliographystyle{apacite}
\bibliography{References}

\end{document}
